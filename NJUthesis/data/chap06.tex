% !Mode:: "Tex:UTF-8"
\chapter{总结与未来工作}
    分布式算法出现在很多领域的应用中,包括电信、分布式信息处理、科学计算和实时进程控制。为这些应用构建分布式系统的重要一步工作就是分布式算法的设计实现和分析。本文主要研究一个针对分布式算法的原型模拟平台$\pozhehao$DAPro。论文前面部分已经详细讨论了我们的工作,下面我们对本文工作做一个总结,并展望未来工作的方向。

    \section{工作总结}
    本文首先简述了分布式算法及其原型实现的概念,介绍了论文的背景。然后概括了其他网络模拟平台和离散事件模拟方面的相关工作,分析了其与本文工作的区别和联系。之后,我们阐述了论文的基础理论知识,对分布式计算模型进行了系统分析。论文的主要部分是DAPro 平台的设计与实现,以及基于DAPro 平台两个分布式算法的实现与分析。总的来看,本文的主要工作包括以下几个方面:
    \begin{itemize}
      \item 详细阐述了DAPro 平台的设计实现。总体概述描述了DAPro 平台的模块结构和运行流程,模块设计部分对平台每个模块的功能和实现进行了详细分析,并以一个DFS 生成树的构建算法为例描述了如何使用DAPro 平台实现一个分布式算法,展示了平台运行算法的流程。
      \item 在原始框架的基础上对DAPro 平台的功能进行了大量扩充。本文在DAPro 平台原始框架的基础上增加了更为丰富的Connector 类型、更多的事件和进程类型、更多的事件动作,并增加了故障生成等模块,提高了系统对于分布式算法的模拟能力。
      \item 基于DAPro 平台实现了分布式的DFS 生成树构建算法和可容错的协商算法。通过具体的实验和对结果的分析,验证了DAPro 平台的正确性和可用性。
    \end{itemize}

    \section{未来工作}
    目前我们已经设计实现了DAPro 平台的系统框架,并在此基础上实现了一些基本的算法,验证了平台的可用性。未来我们的主要工作是要继续优化DAPro,使其成为一个完整而实用的分布式算法模拟平台。具体来说,我们的未来工作包括以下方面:
    \begin{itemize}
      \item 优化DAPro 平台的结构设计。目前DAPro 平台的模块划分(主要是System 模块)尚存在一些疑问,一些模块的实现也并不完整,还有一些功能可以简化。接下来我们将进一步优化DAPro 系统的结构,实现更为清晰的模块划分。
      \item 完善FailureGenerator 模块的设计。目前DAPro 产生故障的逻辑还很原始,是采用直接修改进程状态的方法。为了符合离散事件模拟的思想,我们将增加故障事件,模拟系统中可能出现的各类故障,不仅包括目前所实现的简单崩溃故障,还有拜占庭故障;不仅包括进程发生故障的情况,还应该有信道发生故障的情况。
      \item 更多功能的实现。目前我们已经可以利用DAPro 平台来模拟一些简单的分布式算法,但是这还远远不够。进程间的通信不光可以采用消息传递模型,也可以通过共享变量的方式;平台可以对算法运行数据进行统计;使用XML 文件配置系统拓扑;为平台设计GUI 等都是我们接下来试图达到的目标。
      \item 编辑用户手册。目前DAPro 平台尚无系统的用户手册供读者参考,接下来我们除了继续推进DAPro 的设计实现,还将编辑整理用户手册,提高系统的可读性。
    \end{itemize}