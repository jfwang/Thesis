% !Mode:: "Tex:UTF-8"\chapter{Related work}
\chapter{相关技术}
    本文主要研究基于DAPro 平台的分布式算法实现,与本文相关的工作主要包括其他相关网络模拟平台和离散事件模拟两个方面。下面我们对这两个方面的工作进行简要介绍,并分析其与本文工作的联系和区别。

    \section{网络模拟平台}
    NS2(Network Simulator, version 2)是一款开放源代码的网络仿真软件。它最初为了研究大规模网络以及当前和未来的网络协议交互行为而开发。它为有线和无线网络上的TCP、路由和多播等协议的仿真提供了强有力的支持。NS2 是一个开源项目,所有源代码都开放,任何人可以获得、使用和修改其源代码。正因为如此,世界各地的研究人员每天都在扩展和更新它的功能,为其添加新的协议支持和功能模块。它也是目前网络研究领域应用最广泛的网络仿真软件之一。

    NS2 是一种面向对象的网络仿真器,本质上是一个离散事件模拟器,由UC Berkeley 开发而成。它本身有一个虚拟时钟,所有的仿真都由离散事件驱动的。目前NS2 可以用于仿真各种不同的通信网络,已经实现的一些仿真模块有:网络传输协议,比如TCP 和UDP;业务源流量产生器,比如FTP,Telnet,Web CBR 和VBR;路由队列管理机制,比如Droptail,RED和CBQ;路由算法,比如Dijkstra 等。NS2也为进行局域网的仿真而实现了多播以及一些MAC 子层协议\cite{ns2introduction}。

    然而,NS2 的内容过于庞杂,并且主要面向网络模拟,并非针对分布式算法的设计实现和测试。同时由于需要了解大量相关知识和工具,NS2 对于用户而言难于掌握。本文介绍和使用的DAPro 平台更为精简,专门针对于分布式算法的模拟。用户可以在短时间内掌握平台的基本结构,利用平台进行分布式算法的设计、实现和测试,也可以在遵守DAPro 设计规范的基础上自行扩充其功能,满足自身设计需求。

    \section{离散事件模拟}
    在模拟仿真领域,离散事件模拟将系统的操作建模为基于时间的离散事件序列。每个事件在某个特定瞬间适时发生,并引发系统状态的一个改变\cite{robinson2004simulation}。在连续的事件之间,假定系统不发生任何改变。因此,仿真能够从一个事件适时地转到下一个事件。

    DES 系统通常包含以下几个组件(Component):
    \begin{enumerate}
    \item Clock,时间。系统时间是离散的,可以使用初值为0,不断递增推进的整数序列来模拟。时间为DES 其他组件所共享。比如,事件(Event)会带有时间标签,表明其何时应该被调度。系统引擎(Engine) 会根据被模拟的场景逻辑计算或者判断何时应该产生新的事件,何时应该调度某事件等。
    \item Event,事件。事件是一个DES 系统的核心概念。设计DES 系统,需要明确定义在该系统中哪些行为被视为事件。事件可以具有逻辑上的分类,比如有消息事件(发送消息事件、接收消息事件),延时事件(随机延时、定时延时等)和异常事件(节点失效事件)等。
    \item Eventlist,事件列表。事件带有时间标签,所有的事件通常按照时间先后顺序排列在事件列表中,以等待系统引擎(Engine) 调度执行。事件列表通常使用优先级队列来实现。
    \item Engine,引擎。引擎是DES 系统的控制部件。它负责系统的启动、运行和终止。启动时通常会产生初始事件;系统运行期间不断产生新的事件并调度执行;引擎会判断终止条件是否成立。
    \end{enumerate}

    离散事件模拟的应用十分广泛,DAPro 平台正是基于离散事件模拟设计实现的,系统由一系列离散事件驱动。事实上,离散事件模拟的研究为众多模拟软件提供了理论基础,包括上文中的NS2 网络模拟平台。
